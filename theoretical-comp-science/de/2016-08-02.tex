\documentclass[german,headsepline]{scrartcl}

\usepackage{fontspec}
\usepackage{polyglossia}
\setmainlanguage{german}
\setlength\parindent{0pt}

\usepackage{scrlayer-scrpage}
\pagestyle{scrheadings}
\ihead{\textbf{Aufgaben zur Klausurvorbereitung}\\Einführung in die Theoretische Informatik}
\ohead{Robert Schütz\\2. August 2016}

\usepackage{enumerate}
\usepackage{amsmath, mathtools, amsthm, amsfonts}

\usepackage{exsheets}
\SetupExSheets{headings=block-subtitle}
\SetupExSheets{question/name=Aufgabe}
\SetupExSheets{counter-format=se.qu,counter-within=section}

%\SetupExSheets{solution/print=true}

\theoremstyle{definition}
\newtheorem{ex}{Aufgabe}[section]

\begin{document}
	\section{Maschinen}
	
	\subsection{Turingmaschinen}
	\begin{question}[subtitle={Klausur 2012}]
		Gebe eine 1-Band-Turingmaschine $M$ an,
		die die Funktion $f:\mathbb{N}\times\mathbb{N}\to\mathbb{N}$ mit
		\[f(x,y)=\begin{cases}
			x, &\text{falls $x$ gerade} \\
			y, &\text{falls $x$ ungerade}
		\end{cases}\]
		berechnet.
		
		\textit{Zur Erinnerung}: Die Zahl $n$ wird durch das Unärwort $0^{n+1}$ dargestellt.
	\end{question}
	
	\begin{question}[subtitle={Klausur 2014}]
		Gebe eine 2-Band-Turingmaschine $M$ an, die die durch $f(w)=0^{\#_0(w)}1^{\#_1(w)}$ gegebene Funktion $f:\{0,1\}^*\to\{0,1\}^*$ berechnet.
	\end{question}
	
	\begin{question}[subtitle={Klausur 2015}]
		Gebe einen totalen (deterministischen) 2-Band-Turingakzeptor $M$ an,
		der die Sprache $L=\{w2w\colon w\in\{0,1\}^*\}$ (über dem Alphabet $\{0,1,2\}$) erkennt.
	\end{question}
	
	\begin{question}[subtitle={Blatt 13, 2015}]
		Gebe eine 1-Band-Turingmaschine $M$ an, welche die Menge
		\[A=\{w\in\{0,1\}^*\colon\#_0(w)\text{ gerade}\}\]
		erkennt.
	\end{question}
	
	\begin{question}
		Gebe 3-Band-Turingmaschine $M$ an, die die Funktion $f:\mathbb{N}^2\to\mathbb{N}$ mit
		\[f(x,y)=x\cdot y\]
		berechnet.
	\end{question}
	
	\pagebreak
	\subsection{Registermaschinen}
	\begin{question}[subtitle={Klausur 2015}]
		Die Funktion $f:\mathbb{N}\to\mathbb{N}$ sei durch
		\[f(x)=\begin{cases}
			1, &\text{falls $x$ ungerade} \\
			0, &\text{falls $x$ gerade}
		\end{cases}\]
		definiert.
		Gebe einen Registeroperator $P$ an, der $f$ konservativ berechnet.
	\end{question}
	\begin{solution}
		\[P\equiv[s_1a_2a_3]_1[s_2a_1]_2[s_1s_1a_2a_2]_1[s_3s_2a_1]_3\]
	\end{solution}
	
	\begin{question}[subtitle={Nachklausur 2015}]
		Gebe einen konservativen Registeroperator an,
		der die Gleichheitsrelation erkennt, d.h. die Funktion
		\[c_=(x,y)=1\Leftrightarrow x=y\]
		berechnet.
	\end{question}
	\begin{solution}
		Es gilt
		\[x=y\Leftrightarrow (x\dot{-}y)+(y\dot{-}x)=0.\]
		Diese Darstellung liegt dem folgenden Registeroperator zur Berechnung von $c_=(x,y)$ zugrunde:
		\[P\equiv T_{1\to4,3}T_{1\to6,3}T_{2\to5,3}T_{2\to7,3}[s_4s_7]_7[s_5s_6]_6[a_4s_5]_5a_3[s_3s_4]_4\]
	\end{solution}
	
	\begin{question}[subtitle={Blatt 13, 2015}]
		Gebe eine Registermaschine $R$ an, welche die Funktion $f:\mathbb{N}^2\to\mathbb{N}$ mit
		\[f(x,y)=\begin{cases}
			\frac{x}{3}, &\text{falls $x$ durch drei teilbar} \\
			y, &\text{sonst}
		\end{cases}\]
		berechnet.
	\end{question}
	
	\section{Funktionen}
	\begin{question}
		Zeige nur unter Verwendung der Definition der primitiv rekursiven Funktionen,
		dass folgende Funktionen primitiv rekursiv sind:
		\begin{enumerate}[(a)]
			\item $f:\mathbb{N}\to\mathbb{N}$ mit $f(x)=x^2+2x+1$
			\item $g:\mathbb{N}^2\to\mathbb{N}$ mit $g(x,y)=2^x+2^y$
				\hfill\textit{(Nachklausur 2015)} \\
				Du darfst verwenden, dass die Addition $add(x,y)=x+y$ primitiv rekursiv ist. \\
				\textit{Hinweis}: Zeige zunächst, dass die Funktion $\hat{f}(x)=2^x$ primitiv rekursiv ist.
			\item $h:\mathbb{N}^3\to\mathbb{N}$ mit $h(x,y,z)=3x+z$
		\end{enumerate}
	\end{question}
	
	\begin{question}[subtitle={Nachklausur 2013}]
		Eine natürliche Zahl heißt \emph{Mersenne-Primzahl}, falls sie von der Form $2^n-1$ (für ein $n\in\mathbb{N}$) und eine Primzahl ist.
		
		Zeige, dass die Menge
		\[M=\{x\colon x\text{ ist Mersenne-Primzahl}\}\]
		primitiv rekursiv ist.
		
		Du darfst dabei neben der Definition der primitiv rekursiven Funktionen die aus der Vorlesung bekannten Abschlusseigenschaften sowie die primitive Rekursivität der Multiplikation $\cdot$, der modifizierten Subtraktion $\dot{-}$, der Potenzfunktion $n\mapsto2^n$ und der Gleichheitsrelation verwenden.
	\end{question}
	\begin{solution}
		Es gilt
		\[x\in M\Leftrightarrow(x\neq0~\&~x\neq1~\&~\exists y<x(x=2^y\dot{-}1)~\&~(\forall m<x(\forall n<x(n\cdot m\neq x))).\]
		
		Da die primitiv rekursiven Mengen gegen explizite Definitionen,
		gegen beschränkte Quantoren und gegen die aussagenlogischen Junktoren abgeschlossen sind
		und aufgrund der primitiven Rekursivität der oben genannten Funktionen
		ist $M$ damit primitiv rekursiv.
	\end{solution}
	
	\begin{question}[subtitle={Blatt 13, 2015}]
		Sei $f:\mathbb{N}\to\mathbb{N}^2$ gegeben durch
		\begin{align*}
			f(0) &= 0 \\
			f(x+1) &= \begin{cases}
				f(x)+1, &\text{falls $x$ gerade} \\
				f(x)+2, &\text{falls $x$ ungerade}
			\end{cases}
		\end{align*}
		Zeige, dass $f$ primitiv rekursiv ist.
	\end{question}
	
	\section{Mengen}
	\subsection{Rekursiv aufzählbare Mengen}
	\begin{question}
		Welche der folgenden Aussagen sind wahr?
		Begründe deine Antworten und gebe für falsche Aussagen ein Gegenbeispiel an.
		\begin{enumerate}[(a)]
			\item Jede unendliche rekursive Menge enthält eine nichtrekursive Teilmenge.
			\item Für zwei rekursiv aufzählbare Mengen $A$ und $B$ ist auch $A\setminus B$ rekursiv aufzählbar.
			\item Seien $A$ und $B$ zwei disjunkte rekursiv aufzählbare Mengen, sodass $A\cup B$ rekursiv ist.
			Dann sind auch $A$ und $B$ rekursiv.
			\item Sind sowohl $A$ als auch $\overline{A}$ rekursiv aufzählbar, so gilt $A=_m\overline{A}$.
		\end{enumerate}
		\textit{Hinweis}: Du darfst alle Resultate der Vorlesung inklusive der Church-Turing-These verwenden.
	\end{question}
	\begin{solution}
		\begin{enumerate}[(a)]
			\item Wahr. Für $A\subseteq\mathbb{N}$ unendlich ist die Potenzmenge $\mathcal{P}(A)$,
			d.h. die Menge der Teilmengen von $A$, überabzählbar.
			Die Klasse der rekursiven Mengen ist allerdings abzählbar.
			\item Falsch. Zum Beispiel sind $A=\mathbb{N}$ und $B=K$ laut Vorlesung rekursiv aufzählbar.
			Allerdings ist $A\setminus B=\overline{K}$ nicht rekursiv aufzählbar,
			da sonst nach dem Komplementlemma $K$ rekursiv wäre.
			\item Wahr.
			\item Falsch. Dies gilt nur, falls $A\neq\emptyset,\mathbb{N}$:
			Sind $A$ und $\overline{A}$ rekursiv aufzählbar,
			so sind $A$ und $\overline{A}$ nach dem Komplementlemma rekursiv.
			Laut Vorlesung sind alle rekursiven Mengen ungleich $\emptyset,\mathbb{N}$ $m$-äquivalent,
			was leicht einzusehen ist. \\
			Gilt aber zum Beispiel $A=\emptyset$, so ist $\overline{A}=\mathbb{N}$.
			Allerdings gilt $\mathbb{N}\not\leq_m\emptyset$:
			Aus $\mathbb{N}\leq_m\emptyset$ via $f$, also $x\in\mathbb{N}\Leftrightarrow f(x)\in\emptyset$,
			folgt $f(x)\not\in\mathbb{N}$ für alle $x\in\mathbb{N}$,
			was ein Widerspruch zur Rekursivität von $f$ ist.
		\end{enumerate}
	\end{solution}
	
	\begin{question}
		Die symmetrische Differenz zweier Mengen $A$ und $B$ ist
		\[A\triangle B=(A\setminus B)\cup(B\setminus A)\]
		Zeige:
		\begin{enumerate}[(a)]
			\item Ist $A\triangle B$ endlich, so gilt $A=_mB$.
			\item Die symmetrische Differenz zweier rekursiv aufzählbarer Sprachen ist nicht notwendig rekursiv aufzählbar.
			\item Seien $A$ und $B$ rekursiv aufzählbar.
			Wenn $A\triangle B$ rekursiv ist, dann sind auch $A\setminus B$ und $B\setminus A$ rekursiv.
			%\item Sei $A$ rekursiv. Ist $A\triangle B$ rekursiv aufzählbar, so sind auch $A\setminus B$ und $B\setminus A$ rekursiv aufzählbar.
			%			\item Sei $A$ nicht rekursiv und $B$ rekursiv.
			%				Wenn $A\triangle B$ rekursiv ist, dann ist $A\cap B$ nicht rekursiv.
		\end{enumerate}
		\textit{Hinweis}: Du darfst alle Resultate der Vorlesung inklusive der Church-Turing-These verwenden.
	\end{question}
	
	\begin{question}
		Zeige mithilfe der Diagonalisierungsmethode, dass die Menge
		\[K'=\{e\in\mathbb{N}\colon\varphi(e,x)\downarrow\text{ für alle }x\leq e\}\]
		nicht rekursiv ist.
	\end{question}
	\begin{solution}
		Angenommen, $K'$ sei rekursiv. Dann ist die durch
		\[f(x)=\begin{cases}
			\varphi(x,x)+1, &\text{falls }x\in K' \\
			0, &\text{sonst}
		\end{cases}\]
		definierte Funktion $f:\mathbb{N}\to\mathbb{N}$ rekursiv,
		da $f$ durch eine (nach Annahme) rekursive Fallunterscheidung definiert ist.
		Also existiert ein $e\in\mathbb{N}$ mit $\varphi_e=f$.
		$f$ ist total, also gilt $f(x)\downarrow$ für alle $x\leq e$ und damit
		\[\varphi_e(e)=f(e)=\varphi(e,e)+1.\]
		Dies ist ein Widerspruch. Deshalb kann $K'$ nicht rekursiv sein.
	\end{solution}
	
	\begin{question}[subtitle={Klausur 2015}]
		Beweise oder widerlege die folgenden Aussagen:
		\begin{enumerate}[(a)]
			\item Für alle rekursiv aufzählbaren Mengen $A_0,A_1,A_2$ ist die zugehörige Menge
				\[B=\{x\colon x\text{ ist Element von \emph{mindestens} zwei der Mengen }A_0,A_1,A_2\}\]
				wiederum rekursiv aufzählbar.
			\item Für alle rekursiv aufzählbaren Mengen $A_0,A_1,A_2$ ist die zugehörige Menge
				\[C=\{x\colon x\text{ ist Element von \emph{höchstens} zwei der Mengen }A_0,A_1,A_2\}\]
				wiederum rekursiv aufzählbar.
		\end{enumerate}
		Du darfst hierbei alle Ergebnisse aus der Vorlesung verwenden, nicht aber die Church-Turing-These.
	\end{question}
	
	\begin{question}[subtitle={Blatt 13, 2015}]
		\begin{enumerate}[(a)]
			\item Zeige, dass für eine Menge $A\subseteq\mathbb{N}$ folgende Aussagen äquivalent sind:
				\begin{enumerate}[(i)]
					\item $A$ ist rekursiv.
					\item $A$ und $\overline{A}$ sind rekursiv aufzählbar.
					\item $A$ ist leer oder der Wertebereich einer schwach monotonen, rekursiven Funktion.
						Hierbei heißt $g:\mathbb{N}\to\mathbb{N}$ \emph{schwach monoton},
						falls für alle $x\leq y$ folgt dass $g(x)\leq g(y)$.
					\item $A$ ist endlich oder der Wertebereich einer streng monotonen, rekursiven Funktion.
						Hierbei heißt $g:\mathbb{N}\to\mathbb{N}$ \emph{streng monoton}, falls für alle $x<y$ folgt, dass $g(x)<g(y)$.
				\end{enumerate}
			\item Zeige:
				\begin{enumerate}[(i)]
					\item Ist $A\subseteq\mathbb{N}$ unendlich und rekursiv aufzählbar,
						so ist $A$ der Wertebereich einer totalen, injektiven, rekursiven Funktion.
					\item Ist $A\subseteq\mathbb{N}$ unendlich und rekursiv aufzählbar,
						so enthält $A$ eine unendliche, rekursive Teilmenge $B$.
					\item Ist $A$ rekursiv aufzählbar, so ist $A$ der Definitions- und Wertebereich einer partiell rekursiven Funktion $\psi$.
					
				\end{enumerate}
		\end{enumerate}
	\end{question}
	
	\subsection{Reduktionsmethode}
	\begin{question}[subtitle={Nachklausur 2012}]
		\begin{enumerate}[(a)]
			\item Zeige, dass die Mengen
			\[A=\{\langle x,y\rangle\colon\varphi_x(y)\downarrow\text{ und }\varphi_y(x)\downarrow\}\]
			und
			\[B=\{\langle x,y\rangle\colon\varphi_x(y)\uparrow\text{ oder }\varphi_y(x)\uparrow\}\]
			nicht rekursiv sind.
			\item Sind $A$ und $B$ rekursiv aufzählbar?
		\end{enumerate}
	\end{question}
	\begin{solution}
		\begin{enumerate}[(a)]
			\item Es gilt $K_d\leq_m A$ via $f(x)=\langle x,x\rangle$ da
			\[x\in K_d\Leftrightarrow\varphi_x(x)\downarrow\Leftrightarrow
			\varphi_x(x)\downarrow~\&~\varphi_x(x)\downarrow\Leftrightarrow
			f(x)=\langle x,x\rangle\in A\]
			Da das diagonale Halteproblem nicht rekursiv ist,
			ist also nach dem Reduktionslemma auch $A$ nicht rekursiv.
			Da $B=\overline{A}$ und die rekursiven Mengen gegen Komplement abgeschlossen sind,
			ist somit auch $B$ nicht rekursiv.
			\item Aufgrund des Reduktionslemmas können nicht beide Mengen rekursiv aufzählbar sein.
			$A$ ist als Definitionsbereich der partiell rekursiven Funktion
			\[\psi(\langle x,y\rangle)=\varphi(x,y)+\varphi(y,x)\]
			rekursiv aufzählbar.
			Da eine Menge rekursiv ist, falls die Menge selbst und ihr Komplement rekursiv aufzählbar sind,
			kann also $B$ nicht rekursiv aufzählbar sein.
		\end{enumerate}
	\end{solution}
	
	\begin{question}
		Seien FIN und INF die Indizes der endlichen bzw. unendlichen rekursiv aufzählbaren Mengen:
		\begin{align*}
			\text{FIN} &= \{e\in\mathbb{N}\colon W_e\text{ ist endlich}\} \\
			\text{INF} &= \{e\in\mathbb{N}\colon W_e\text{ ist unendlich}\}
		\end{align*}
		Sei TOT die Menge der Indizes aller rekursiven Funktionen:
			\[\text{TOT}=\{e\in\mathbb{N}\colon W_e=\mathbb{N}\}\]
		\textit{Hinweis}: Wie in der Vorlesung definiert,
		ist $W_e$ der Definitionsbereich der $e$-ten partiell rekursiven Funktion,
		also $W_e=Db(\varphi_e)=\{x\in\mathbb{N}\colon\varphi_e(x)\downarrow\}$.
		\begin{enumerate}[(a)]
			\item Zeige, dass die Menge
				$\text{FIN}\oplus\text{INF}=\{2e\colon e\in\text{FIN}\}\cup\{2e+1\colon e\in\text{INF}\}$
				nicht rekursiv aufzählbar ist.
			\item Zeige, dass das Komplement von TOT auf FIN many-one-reduzierbar ist,
				also dass $\overline{\text{TOT}}\leq_m\text{FIN}$ gilt.
			\item Zeige, dass FIN nicht rekursiv aufzählbar ist. \\
				\textit{Hinweis}: Entweder zeigst du dies direkt,
				oder du zeigst durch geeignete $m$-Reduzierung,
				dass $\overline{\text{TOT}}$ nicht rekursiv aufzählbar ist.
		\end{enumerate}
	\end{question}
	\begin{solution}
		\begin{enumerate}[(a)]
			\item FIN und INF sind als nichttriviale Indexmengen nach dem Satz von Rice nicht rekusiv.
				Da aber $\text{FIN}=\overline{\text{INF}}$ gilt,
				muss nach dem Komplementlemma zumindest eine der Mengen nicht rekursiv aufzählbar sein.
				Aus $\text{FIN}\leq_m\text{FIN}\oplus\text{INF}$ und $\text{INF}\leq_m\text{FIN}\oplus\text{INF}$
				folgern wir mithilfe der Abgeschlossenheit der rekursiv aufzählbaren Mengen nach unten gegen $m$-Reduzierung,
				dass $\text{FIN}\oplus\text{INF}$ nicht rekusiv aufzählbar ist.
			\item Definiere die zweistellige partiell rekursive Funktion $\psi$ durch
				\[\psi(e,x)=\sum_{n=0}^{x}\varphi_e(n).\]
				Dann gibt es, da $\varphi$ eine Gödelnummerierung ist,
				eine rekursive Übersetzungsfunktion $f$ von $\psi$ nach $\varphi$,
				d.h. $\psi(e,x)=\varphi_{f(e)}(x)$ für alle $x\in\mathbb{N}$.
				Somit gilt \begin{align*}
					e\in\overline{\text{TOT}}
					&\Rightarrow e\not\in\text{TOT} \\
					&\Rightarrow\varphi_e(x_0)\uparrow\text{ für ein }x_0\in\mathbb{N} \\
					&\Rightarrow\psi_e(x)\uparrow\text{ für alle }x\geq x_0 \\
					&\Rightarrow W_{f(e)}=Db(\varphi_{f(e)})=Db(\psi_e)\text{ ist endlich} \\
					&\Rightarrow f(e)\in\text{FIN} \\
				\end{align*} und \begin{align*}
					e\not\in\overline{\text{TOT}}
					&\Rightarrow e\in\text{TOT} \\
					&\Rightarrow\varphi_e(x)\downarrow\text{ für alle }x\in\mathbb{N} \\
					&\Rightarrow\psi_e(x)\downarrow\text{ für alle }x\in\mathbb{N} \\
					&\Rightarrow W_{f(e)}=Db(\varphi_{f(e)})=Db(\psi_e)=\mathbb{N} \text{ ist unendlich} \\
					&\Rightarrow f(e)\in\text{INF}=\overline{\text{FIN}},
				\end{align*}
				d.h. $e\in\overline{\text{TOT}}\Leftrightarrow f(e)\in\text{FIN}$
				und deshalb $\overline{\text{TOT}}\leq_m\text{FIN}$ via $f$.
			\item Es genügt, zu zeigen, dass $\overline{\text{TOT}}$ nicht rekursiv aufzählbar ist,
				denn dann folgt aus der Abgeschlossenheit der rekursiv aufzählbaren Mengen nach unten gegen $m$-Reduzierung,
				dass FIN nicht rekursiv aufzählbar ist.
				Definiere also die zweistellige partiell rekursive Funktion $\theta$ durch
				\[\theta(e,x)=\varphi_e(e).\]
				Dann gibt es, da $\varphi$ eine Gödelnummerierung ist,
				eine rekursive Übersetzungsfunktion $g$ von $\theta$ nach $\varphi$,
				d.h. $\theta(e,x)=\varphi_{g(e)}(x)$ für alle $x\in\mathbb{N}$.
				Somit gilt \[
					e\in K_d
					\Leftrightarrow\varphi_e(e)\downarrow
					\Leftrightarrow\theta_e\text{ total}
					\Leftrightarrow\varphi_{g(e)}\text{ total}
					\Leftrightarrow g(e)\in\text{TOT},
				\]
				d.h. $K_d\leq_m\text{TOT}$ bzw. $\overline{K_d}\leq_m\overline{\text{TOT}}$ via $g$.
				Aufgrund der Transitivität von $\leq_m$ folgt $\overline{K_d}\leq_m\text{FIN}$.
				Also wäre mit FIN auch $\overline{K_d}$ rekursiv aufzählbar,
				da die rekursiv aufzählbaren Mengen nach unten gegen $\leq_m$ abgeschlossen sind.
				Dies ergäbe allerdings einen Widerspruch,
				denn dann wären $K_d$ (laut Vorlesung) und $\overline{K_d}$ (nach Annahme) rekursiv aufzählbar und somit wäre $K_d$ rekursiv.
		\end{enumerate}
	\end{solution}
	
	\begin{question}[subtitle={Klausur 2012}]
		Sei
		\[\text{Id}=\{e\in\mathbb{N}\colon\forall n~(\varphi_e(n)=n)\}.\]
		\begin{enumerate}[(a)]
			\item Zeige, dass das Halteproblem $m$-reduzierbar auf Id ist, das heißt, dass $K\leq_m\text{Id}$ gilt.
			\item Gebe für die folgenden Aussagen jeweils an, ob sie wahr oder falsch sind.
				Begründe deine Antworten!
				\begin{enumerate}[(i)]
					\item Id ist rekursiv.
					\item $\overline{\text{Id}}$ ist rekursiv.
					\item $\overline{\text{Id}}$ ist rekursiv aufzählbar.
				\end{enumerate}
		\end{enumerate}
	\end{question}
	\begin{solution}
		\begin{enumerate}[(a)]
			\item Definiere die zweistellige partiell rekursive Funktion $\psi$ durch
				\[\psi(\langle e,x\rangle,y)=0\cdot\varphi_e(x)+y.\]
				Dann gibt es, da $\varphi$ eine Gödelnummerierung ist,
				eine rekursive Übersetzungsfunktion $f$ von $\psi$ nach $\varphi$,
				d.h. $\psi(\langle e,x\rangle,y)=\varphi_{f(\langle e,x\rangle)}(y)$ für alle $y\in\mathbb{N}$.
				Somit gilt
				\begin{align*}
					\langle e,x\rangle\in K &\Leftrightarrow \varphi_e(x)\downarrow \\
					&\Leftrightarrow \psi(\langle e,x\rangle,y)=y\text{ für alle }y\in\mathbb{N} \\
					&\Leftrightarrow \varphi_{f(\langle e,x\rangle)}(y)=y\text{ für alle }y\in\mathbb{N} \\
					&\Leftrightarrow f(\langle e,x\rangle)\in\text{Id},
				\end{align*}
				d.h. $K\leq_m\text{Id}$ via $f$.
			\item \begin{enumerate}[(i)]
				\item Falsch. Id ist als nichttriviale Indexmenge nach dem Satz von Rice nicht rekursiv.
					Alternativ kann man argumentieren, dass $K$ nicht rekursiv ist und somit nach dem Reduktionslemma Id auch nicht.
				\item Falsch. Eine Menge ist genau dann rekursiv, wenn ihr Komplement rekursiv ist.
				\item Aus $K\leq_m\text{Id}$ via $f$ folgt $\overline{K}\leq_m\overline{\text{Id}}$ via $f$.
					Also wäre mit $\overline{\text{Id}}$ auch $\overline{K}$ rekursiv aufzählbar,
					da die rekursiv aufzählbaren Mengen nach unten gegen $\leq_m$ abgeschlossen sind.
					Dies ergäbe allerdings einen Widerspruch,
					denn dann wären $K$ (laut Votlesung) und $\overline{K}$ (nach Annahme) rekursiv aufzählbar und somit wäre $K$ rekursiv.
			\end{enumerate}
		\end{enumerate}
	\end{solution}
	
	\pagebreak
	\begin{question}[subtitle={Blatt 13, 2015}]
		\begin{enumerate}[(a)]
			\item Sei $W_e=Db(\varphi_e)$ die $e$-te rekursiv aufzählbare Menge
				und $W=\{e\in\mathbb{N}\colon0,1\in W_e\}$.
				Zeige $K_d\leq_mW$, indem du eine rekursive Funktion $f:\mathbb{N}\to\mathbb{N}$ angibst mit
				\[\forall e\colon e\in K_d\Leftrightarrow f(e)\in W\]
				Ist $W$ rekursiv aufzählbar?
			\item Folgere aus (a), dass $W$ vollständig für die Klasse der rekursiv aufzählbaren Mengen ist.
			\item Sei nun $I=\{e\in\mathbb{N}\colon W_e\text{ ist nichttriviale Indexmenge}\}$.
				Ist $I$ eine Idexmenge? Ist sie nichttrivial?
			\item Wahr oder falsch?
				Begründe deine Antwort und gebe gegebenenfalls Gegenbeispiele an!
				\begin{enumerate}[(i)]
					\item Ist $B$ rekursiv aufzählbar und $A\leq_m B$, so ist auch $A$ rekursiv aufzählbar.
					\item Man kann den kleinsten Index einer partiell rekursiven Funktion bestimmen.
						Genauer: Die Funktion $min$,
						die jedem Index $e$ die kleinste Zahl $e'$ mit $\varphi_e=\varphi_{e'}$ zuordnet, ist rekursiv
					\item Es gibt eine abzählbare Klasse $C$, die keine vollständige Menge besitzt.
				\end{enumerate}
		\end{enumerate}
	\end{question}
	
	\section{Formale Sprachen}
	\begin{question}
		Zeige, dass die folgenden Sprachen nicht kontextfrei sind:
		\begin{align*}
			L_1 &= \{w\in\{a,b,c\}^*\colon \#_a(w)=\#_b(w)=\#_c(w)\} & \textit{(Klausur 2014)} \\
			L_2 &= \{a^nb^{2n}c^{3n}\colon n\geq1\} & \textit{(Klausur 2015)} \\
			L_3 &= \{0^{n^2}\colon n\geq0\} & \textit{(Nachklausur 2015)}
		\end{align*}
	\end{question}
	
	\begin{question} % http://algo2.iti.kit.edu/3126.php
		Wahr oder falsch?
		\begin{enumerate}[(a)]
			\item Die Sprache $L_1=\{vw\colon v,w\in\{0,1\}^*,\vert v\vert=\vert w\vert\}$ ist regulär.
			\item Die Sprache $L_2=\{w_0\circ\ldots\circ w_n\colon n\in\mathbb{N},\vert w_k\vert=k~\forall k\in\mathbb{N}\}$ ist kontextfrei.
			\item Sei $L\subseteq\{0,1\}^*$ eine Sprache mit $\vert L\vert=n$ für ein $n\in\mathbb{N}$. Dann ist $L$ regulär.
		\end{enumerate}
	\end{question}
	\begin{solution}
		\begin{enumerate}[(a)]
			\item Wahr. $L_1$ wird von der Grammatik $G=(\{S,T\},\{0,1\},P,S)$ mit den Regeln
				\[P=\{S\rightarrow0T~|~1T~|~\lambda,~T\rightarrow0S~|~1S\}\]
				erzeugt, denn
				\[L=\{w\in\{0,1\}^*\colon\vert w\vert=2n\text{ für ein }n\in\mathbb{N}\}.\]
			\item Falsch. \textbf{TODO}: Pumpinglemma anwenden
			\item Wahr. $L$ wird von der regulären Grammatik $G=(\{S\},\{0,1\},P,S)$ mit den Regeln
				\[P=\{S\to w\colon w\in L\}\]
				erzeugt. Dabei ist $P$ wegen $\vert P\vert=\vert L\vert=n$ endlich.
		\end{enumerate}
	\end{solution}
	
	\begin{question}
		Zeige, dass die Sprache
		\[L=\{w\in\{0,1\}^*\colon w=Bin(n)\text{ für ein }n\in\mathbb{N}\}\]
		der Binärzahlen regulär ist.
	\end{question}
	\begin{solution}
		$L$ wird von der Grammatik $G=(\{S,A\},\{0,1\},P,S)$ mit den Regeln
		\[P=\{S\rightarrow0~|~1A,~A\rightarrow0A~|~1A~|~\lambda\}\]
		ezeugt, denn
		\[L=\{0\}\cup\{1w\colon w\in\{0,1\}^*\}\text{.}\]
	\end{solution}
	
	\begin{question}
		Sei $L$ die Sprache über dem Alphabet $\{a,b,c\}$, die genau die Wörter enthält,
		in denen das Teilwort $abc$ genau einmal vorkommt.
		\begin{enumerate}[(a)]
			\item Gebe eine rechtslineare Grammatik an, die $L$ erzeugt.
			\item Gebe einen deterministischen endlichen Automaten an, der $L$ erkennt.
		\end{enumerate}
	\end{question}
	\begin{solution}
		\begin{enumerate}[(a)]
			\item $L$ wird von der Grammatik $G=(\{S,A,B,C,D,E\},\{a,b,c\},P,S)$ mit den Regeln
				\begin{align*}
					P=\{&S\rightarrow aA~|~bS~|~cS, \\
					&A\rightarrow aA~|~bB~|~cS, \\
					&B\rightarrow aA~|~bS~|~cC, \\
					&C\rightarrow aD~|~bC~|~cC~|~\lambda, \\
					&D\rightarrow aD~|~bE~|~cC~|~\lambda, \\
					&E\rightarrow aD~|~bC~|~\lambda\}
				\end{align*}
				erzeugt.
		\end{enumerate}
	\end{solution}
	
	\begin{question}
		Sei $L$ die reguläre Sprache
		\[L=\{w\in\{0,1,2\}^*\colon\#_0(w)=0\text{ oder }\#_1(w)=0\text{ oder }\#_2(w)=0\}.\]
		Gebe einen deterministischen endlichen Automaten an, der $L$ erkennt.
	\end{question}
	\begin{solution}
		Zunächst beobachtet man, dass
		\[L=\{0,1\}^*\cup\{0,2\}^*\cup\{1,2\}^*\]
		gilt.
		
		\textbf{TODO}
	\end{solution}
	
	\begin{question}[subtitle={Klausur 2012}]
		Es sei $G=(\{S,X,Y,Z\},\{a,b\},P,S)$ die kontextfreie Grammatik mit der Regelmenge
		\[P=\{S\to XY,~X\to ZXb~|~\lambda,~Y\to bY~|~b,~Z\to a~|~\lambda\}.\]
		\begin{enumerate}[(a)]
			\item Gebe die von $G$ erzeugte Sprache $L(G)$ an.
			\item Gebe ein Wort $w\in L(G)$ minimaler Länge an,
				das mehr als einen Herleitungsbaum besitzt,
				und zeichne zwei verschiedene zugehörigen Herleitungsbäume.
			\item Gebe eine kontextfreie Grammatik in Chomsky-Normalform an, die $L(G)$ erzeugt.
		\end{enumerate}
	\end{question}
\end{document}
