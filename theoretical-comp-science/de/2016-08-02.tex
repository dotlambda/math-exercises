\documentclass[german,headsepline]{scrartcl}

\usepackage{fontspec}
\usepackage{polyglossia}
\setmainlanguage{german}
\setlength\parindent{0pt}

\usepackage{scrlayer-scrpage}
\pagestyle{scrheadings}
\ihead{\textbf{Aufgaben zur Klausurvorbereitung}\\Einführung in die Theoretische Informatik}
\ohead{Robert Schütz\\2. August 2016}

\usepackage{enumerate}
\usepackage{amsmath, mathtools, amsfonts}

\usepackage{exsheets}
\SetupExSheets{headings=block-subtitle}
\SetupExSheets{question/name=Aufgabe}
\SetupExSheets{counter-format=se.qu,counter-within=section}

\SetupExSheets{solution/print=true}

\begin{document}	
	\section{Turingmaschinen}
	\begin{question}[subtitle={Klausur 2012}]
		Gebe eine 1-Band-Turingmaschine $M$ an,
		die die Funktion $f:\mathbb{N}\times\mathbb{N}\to\mathbb{N}$ mit
		\[f(x,y)=\begin{cases}
			x, &\text{falls $x$ gerade} \\
			y, &\text{falls $x$ ungerade}
		\end{cases}\]
		berechnet.
		
		\textit{Zur Erinnerung}: Die Zahl $n$ wird durch das Unärwort $0^{n+1}$ dargestellt.
	\end{question}
	
	\begin{question}[subtitle={Klausur 2014}]
		Gebe eine 2-Band-Turingmaschine $M$ an, die die durch $f(w)=0^{\#_0(w)}1^{\#_1(w)}$ gegebene Funktion $f:\{0,1\}^*\to\{0,1\}^*$ berechnet.
	\end{question}
	
	\begin{question}[subtitle={Klausur 2015}]
		Gebe einen totalen (deterministischen) 2-Band-Turingakzeptor $M$ an,
		der die Sprache $L=\{w2w\colon w\in\{0,1\}^*\}$ (über dem Alphabet $\{0,1,2\}$) erkennt.
	\end{question}
	
	\begin{question}[subtitle={Blatt 13, 2015}]
		Gebe eine 1-Band-Turingmaschine $M$ an, welche die Menge
		\[A=\{w\in\{0,1\}^*\colon\#_0(w)\text{ gerade}\}\]
		erkennt.
	\end{question}
	
	\begin{question}
		Gebe eine 3-Band-Turingmaschine $M$ an, die die Funktion $f:\mathbb{N}^2\to\mathbb{N}$ mit
		\[f(x,y)=x\cdot y\]
		berechnet.
	\end{question}
	
	\section{Registermaschinen}
	\begin{question}[subtitle={Klausur 2015}]
		Die Funktion $f:\mathbb{N}\to\mathbb{N}$ sei durch
		\[f(x)=\begin{cases}
			1, &\text{falls $x$ ungerade} \\
			0, &\text{falls $x$ gerade}
		\end{cases}\]
		definiert.
		Gebe einen Registeroperator $P$ an, der $f$ konservativ berechnet.
	\end{question}
	\begin{solution}
		\[P\equiv[s_1a_2a_3]_1[s_2a_1]_2[s_1s_1a_2a_2]_1[s_3s_2a_1]_3\]
	\end{solution}
	
	\begin{question}[subtitle={Nachklausur 2015}]
		Gebe einen konservativen Registeroperator an,
		der die Gleichheitsrelation erkennt, d.h. die Funktion
		\[c_=(x,y)=1\Leftrightarrow x=y\]
		berechnet.
	\end{question}
	\begin{solution}
		Es gilt
		\[x=y\Leftrightarrow (x\dot{-}y)+(y\dot{-}x)=0.\]
		Diese Darstellung liegt dem folgenden Registeroperator zur Berechnung von $c_=(x,y)$ zugrunde:
		\[P\equiv T_{1\to4,3}T_{1\to6,3}T_{2\to5,3}T_{2\to7,3}[s_4s_7]_7[s_5s_6]_6[a_4s_5]_5a_3[s_3s_4]_4\]
	\end{solution}
	
	\begin{question}[subtitle={Blatt 13, 2015}]
		Gebe eine Registermaschine $R$ an, welche die Funktion $f:\mathbb{N}^2\to\mathbb{N}$ mit
		\[f(x,y)=\begin{cases}
			\frac{x}{3}, &\text{falls $x$ durch drei teilbar} \\
			y, &\text{sonst}
		\end{cases}\]
		berechnet.
	\end{question}
	
	\section{Primitiv rekursive Funktionen}
	\begin{question}
		Zeige nur unter Verwendung der Definition der primitiv rekursiven Funktionen,
		dass folgende Funktionen primitiv rekursiv sind:
		\begin{enumerate}[(a)]
			\item $f:\mathbb{N}\to\mathbb{N}$ mit $f(x)=x^2+2x+1$
			\item $g:\mathbb{N}^2\to\mathbb{N}$ mit $g(x,y)=2^x+2^y$
				\hfill\textit{(Nachklausur 2015)} \\
				Du darfst verwenden, dass die Addition $add(x,y)=x+y$ primitiv rekursiv ist. \\
				\textit{Hinweis}: Zeige zunächst, dass die Funktion $\hat{f}(x)=2^x$ primitiv rekursiv ist.
			\item $h:\mathbb{N}^3\to\mathbb{N}$ mit $h(x,y,z)=3x+z$
		\end{enumerate}
	\end{question}
	
	\begin{question}[subtitle={Nachklausur 2013}]
		Eine natürliche Zahl heißt \emph{Mersenne-Primzahl}, falls sie von der Form $2^n-1$ (für ein $n\in\mathbb{N}$) und eine Primzahl ist.
		
		Zeige, dass die Menge
		\[M=\{x\colon x\text{ ist Mersenne-Primzahl}\}\]
		primitiv rekursiv ist.
		
		Du darfst dabei neben der Definition der primitiv rekursiven Funktionen
		die aus der Vorlesung bekannten Abschlusseigenschaften
		sowie die primitive Rekursivität der Multiplikation $\cdot$,
		der modifizierten Subtraktion $\dot{-}$,
		der Potenzfunktion $n\mapsto2^n$ und der Gleichheitsrelation verwenden.
	\end{question}
	\begin{solution}
		Es gilt
		\[x\in M\Leftrightarrow(x\neq0~\&~x\neq1~\&~\exists y<x(x=2^y\dot{-}1)~\&~(\forall m<x(\forall n<x(n\cdot m\neq x))).\]
		
		Da die primitiv rekursiven Mengen gegen explizite Definitionen,
		gegen beschränkte Quantoren und gegen die aussagenlogischen Junktoren abgeschlossen sind
		und aufgrund der primitiven Rekursivität der oben genannten Funktionen
		ist $M$ damit primitiv rekursiv.
	\end{solution}
	
	\begin{question}[subtitle={Blatt 13, 2015}]
		Sei $f:\mathbb{N}\to\mathbb{N}^2$ gegeben durch
		\begin{align*}
			f(0) &= 0 \\
			f(x+1) &= \begin{cases}
				f(x)+1, &\text{falls $x$ gerade} \\
				f(x)+2, &\text{falls $x$ ungerade}
			\end{cases}
		\end{align*}
		Zeige, dass $f$ primitiv rekursiv ist.
	\end{question}
	\begin{solution}
		Wegen
		\begin{align*}
			c_{\text{ungerade}}(0) &= 0 \\
			c_{\text{ungerade}}(x+1) &= 1\dot{-}f(x)
		\end{align*}
		ist die charakteristische Funktion der ungeraden Zahlen primitiv rekursiv.
		Wegen
		\begin{align*}
			f(0) &= 0 \\
			f(x+1) &= f(x) + c_{ungerade}(x) + 1
		\end{align*}
		ist dann auch $f$ primitiv rekursiv.
	\end{solution}
	
	\begin{question}[subtitle={Klausur 2009}]
		Die Funktion $f:\mathbb{N}\to\mathbb{N}$ sei durch
		\[f(x)=\begin{cases}
			\sqrt{x}, &\text{falls }\sqrt{x}\in\mathbb{N} \\
			0, &\text{sonst}
		\end{cases}\]
		definiert. Zeige, dass $f$ primitiv rekursiv ist.
		(Du darfst hierbei alle Ergebnisse über die primitiv rekursiven Funktionen aus der Vorlesung verwenden.)
	\end{question}
	\begin{solution}
		Wegen
		\[f(x)=\mu y\leq x(y\cdot y=x),\]
		der Abgeschlossenheit von F(PRIM) gegen den beschränkten $\mu$-Operator
		und weil die Multiplikation primitiv rekursiv ist,
		ist $f$ primitiv rekursiv.
	\end{solution}
	
	\begin{question}[subtitle={Klausur 2009}]
		Die Funktion $f:\mathbb{N}\to\mathbb{N}$ sei primitiv rekursiv und $g(n)$ sei die Funktion,
		die die Anzahl der geraden Nullstellen von $f$ strikt unterhalb von $n$ berechnet, d.h.
		\[g(n)=\vert\{m\colon m<n~\&~m\text{ gerade}~\&~f(m)=0\}\vert.\]
		Zeige, dass $g$ ebenfalls primitiv rekursiv ist.
		(Du darfst hierbei alle Ergebnisse über die primitiv rekursiven Funktionen aus der Vorlesung verwenden.)
	\end{question}
	\begin{solution}
		Wegen
		\begin{align*}
			g(0) &= 0 \\
			g(n+1) &= \begin{cases}
				g(n)+\overline{sg}(f(n)), &\text{falls }\exists m\leq n~(m+m=n) \\
				g(n) &\text{sonst}
			\end{cases}
		\end{align*}
		ist $g$ primitiv rekursiv.
	\end{solution}
	
	\section{Rekursiv aufzählbare Mengen}
	\begin{question}
		Welche der folgenden Aussagen sind wahr?
		Begründe deine Antworten und gebe für falsche Aussagen ein Gegenbeispiel an.
		\begin{enumerate}[(a)]
			\item Jede unendliche rekursive Menge enthält eine nichtrekursive Teilmenge.
			\item Für zwei rekursiv aufzählbare Mengen $A$ und $B$ ist auch $A\setminus B$ rekursiv aufzählbar.
			\item Seien $A$ und $B$ zwei disjunkte rekursiv aufzählbare Mengen, sodass $A\cup B$ rekursiv ist.
			Dann sind auch $A$ und $B$ rekursiv.
			\item Sind sowohl $A$ als auch $\overline{A}$ rekursiv aufzählbar, so gilt $A=_m\overline{A}$.
		\end{enumerate}
		\textit{Hinweis}: Du darfst alle Resultate der Vorlesung inklusive der Church-Turing-These verwenden.
	\end{question}
	\begin{solution}
		\begin{enumerate}[(a)]
			\item Wahr. Für $A\subseteq\mathbb{N}$ unendlich ist die Potenzmenge $\mathcal{P}(A)$,
				d.h. die Menge der Teilmengen von $A$, überabzählbar.
				Die Klasse der rekursiven Mengen ist allerdings abzählbar.
			\item Falsch. Zum Beispiel sind $A=\mathbb{N}$ und $B=K$ laut Vorlesung rekursiv aufzählbar.
				Allerdings ist $A\setminus B=\overline{K}$ nicht rekursiv aufzählbar,
			da sonst nach dem Komplementlemma $K$ rekursiv wäre.
			\item Wahr. Um $x\in A$ zu entscheiden, prüfen wir zunächst, ob $x$ in $A\cup B$ liegt.
				Ist dies der Fall, zählen wir gleichzeitig $A$ und $B$ auf.
				Da $x\in A\cup B$ und $A\cap B=\emptyset$ gilt, liegt $x$ in genau einer der Mengen $A$ und $B$.
				Falls $x$ vom Aufzählungsverfahren für $A$ aufgezählt wird, liegt $x$ in $A$.
				Wird $x$ von dem für $B$ aufgezählt, nicht. Einer dieser Fälle tritt immer ein.
			\item Falsch. Dies gilt nur, falls $A\neq\emptyset,\mathbb{N}$:
				Sind $A$ und $\overline{A}$ rekursiv aufzählbar,
				so sind $A$ und $\overline{A}$ nach dem Komplementlemma rekursiv.
				Laut Vorlesung sind alle rekursiven Mengen ungleich $\emptyset,\mathbb{N}$ $m$-äquivalent,
				was leicht einzusehen ist. \\
				Gilt aber zum Beispiel $A=\emptyset$, so ist $\overline{A}=\mathbb{N}$.
				Allerdings gilt $\mathbb{N}\not\leq_m\emptyset$:
				Aus $\mathbb{N}\leq_m\emptyset$ via $f$, also $x\in\mathbb{N}\Leftrightarrow f(x)\in\emptyset$,
				folgt $f(x)\not\in\mathbb{N}$ für alle $x\in\mathbb{N}$,
				was ein Widerspruch zur Rekursivität von $f$ ist.
		\end{enumerate}
	\end{solution}
	
	\begin{question}
		Die symmetrische Differenz zweier Mengen $A$ und $B$ ist
		\[A\triangle B=(A\setminus B)\cup(B\setminus A)\]
		Zeige:
		\begin{enumerate}[(a)]
			\item Ist $A\triangle B$ endlich und $A,B\neq\emptyset,\mathbb{N}$, so gilt $A=_mB$.
				
				\textit{Bemerkung}: Für $A\triangle B$ endlich schreibt man auch $A=^*B$.
			\item Die symmetrische Differenz zweier rekursiv aufzählbarer Mengen ist nicht notwendig rekursiv aufzählbar.
			\item Seien $A$ und $B$ rekursiv aufzählbar.
				Wenn $A\triangle B$ rekursiv ist, dann sind auch $A\setminus B$ und $B\setminus A$ rekursiv.
			%\item Sei $A$ rekursiv. Ist $A\triangle B$ rekursiv aufzählbar, so sind auch $A\setminus B$ und $B\setminus A$ rekursiv aufzählbar.
			%			\item Sei $A$ nicht rekursiv und $B$ rekursiv.
			%				Wenn $A\triangle B$ rekursiv ist, dann ist $A\cap B$ nicht rekursiv.
		\end{enumerate}
		\textit{Hinweis}: Du darfst alle Resultate der Vorlesung inklusive der Church-Turing-These verwenden.
	\end{question}
	\begin{solution}
		\begin{enumerate}[(a)]
			\item Ist $A\triangle B$ endlich, so sind auch $A\setminus B$ und $B\setminus A$ endlich.
				Sei $A\setminus B=\{x_1,\dots,x_n\}$ und $B\setminus A=\{y_1,\dots,y_m\}$.
				Dann gilt $A\leq_mB$ via $f$ mit
				\[f(x)=\begin{cases}
					\overline{b}, &\text{falls }x\in\{x_1,\dots,x_n\} \\
					b, &\text{falls }x\in\{y_1,\dots,y_m\} \\
					x, &\text{sonst,}
				\end{cases}\]
				wobei $b\in B$ und $\overline{b}\notin B$ gilt.
				$f$ ist rekursiv, weil endliche Mengen rekursiv sind. $B\leq_mA$ analog.
			\item $K$ und $\mathbb{N}$ sind rekursiv aufzählbar,
				aber $K\triangle\mathbb{N}=\overline{K}$ ist nicht rekursiv aufzählbar.
			\item Um $x\in A\setminus B$ zu entscheiden, prüfe zunächst, ob $x\in A\triangle B$ gilt.
				Ist dies der Fall, so zählen wir gleichzeitig $A$ und $B$ auf, bis $x$ auftaucht.
				Da $x$ in $A\triangle B$ liegt, muss $x$ in einer der beiden Mengen liegen.
				Gilt $x\in B$, so liegt $x$ in $A\setminus B$. Ansonsten nicht.
		\end{enumerate}
	\end{solution}
	
	\begin{question}
		Zeige mithilfe der Diagonalisierungsmethode, dass die Menge
		\[K'=\{e\in\mathbb{N}\colon\varphi(e,x)\downarrow\text{ für alle }x\leq e\}\]
		nicht rekursiv ist.
	\end{question}
	\begin{solution}
		Angenommen, $K'$ sei rekursiv. Dann ist die durch
		\[f(x)=\begin{cases}
			\varphi(x,x)+1, &\text{falls }x\in K' \\
			0, &\text{sonst}
		\end{cases}\]
		definierte Funktion $f:\mathbb{N}\to\mathbb{N}$ rekursiv,
		da $f$ durch eine (nach Annahme) rekursive Fallunterscheidung definiert ist.
		Also existiert ein $e\in\mathbb{N}$ mit $\varphi_e=f$.
		$f$ ist total, also gilt $f(x)\downarrow$ für alle $x\leq e$, deshalb $e\in K'$ und somit
		\[\varphi_e(e)=f(e)=\varphi(e,e)+1.\]
		Dies ist ein Widerspruch. Deshalb kann $K'$ nicht rekursiv sein.
	\end{solution}
	
	\begin{question}[subtitle={Klausur 2015}]
		Beweise oder widerlege die folgenden Aussagen:
		\begin{enumerate}[(a)]
			\item Für alle rekursiv aufzählbaren Mengen $A_0,A_1,A_2$ ist die zugehörige Menge
				\[B=\{x\colon x\text{ ist Element von \emph{mindestens} zwei der Mengen }A_0,A_1,A_2\}\]
				wiederum rekursiv aufzählbar.
			\item Für alle rekursiv aufzählbaren Mengen $A_0,A_1,A_2$ ist die zugehörige Menge
				\[C=\{x\colon x\text{ ist Element von \emph{höchstens} zwei der Mengen }A_0,A_1,A_2\}\]
				wiederum rekursiv aufzählbar.
		\end{enumerate}
		Du darfst hierbei alle Ergebnisse aus der Vorlesung verwenden, nicht aber die Church-Turing-These.
	\end{question}
	
	\begin{question}[subtitle={Blatt 13, 2015}]
		\begin{enumerate}[(a)]
			\item Zeige, dass für eine Menge $A\subseteq\mathbb{N}$ folgende Aussagen äquivalent sind:
				\begin{enumerate}[(i)]
					\item $A$ ist rekursiv.
					\item $A$ und $\overline{A}$ sind rekursiv aufzählbar.
					\item $A$ ist leer oder der Wertebereich einer schwach monotonen, rekursiven Funktion.
						Hierbei heißt $g:\mathbb{N}\to\mathbb{N}$ \emph{schwach monoton},
						falls für alle $x\leq y$ folgt dass $g(x)\leq g(y)$.
					\item $A$ ist endlich oder der Wertebereich einer streng monotonen, rekursiven Funktion.
						Hierbei heißt $g:\mathbb{N}\to\mathbb{N}$ \emph{streng monoton}, falls für alle $x<y$ folgt, dass $g(x)<g(y)$.
				\end{enumerate}
			\item Zeige:
				\begin{enumerate}[(i)]
					\item Ist $A\subseteq\mathbb{N}$ unendlich und rekursiv aufzählbar,
						so ist $A$ der Wertebereich einer totalen, injektiven, rekursiven Funktion.
					\item Ist $A\subseteq\mathbb{N}$ unendlich und rekursiv aufzählbar,
						so enthält $A$ eine unendliche, rekursive Teilmenge $B$.
				\end{enumerate}
		\end{enumerate}
	\end{question}
	\begin{solution}
		\begin{enumerate}[(a)]
			\item Wir zeigen dies per Ringschluss:
				\begin{enumerate}[\hspace{45pt}]
					\item[(i)$\Rightarrow$(iv)] Wir definieren
						\begin{align*}
							g(0) &= \mu x(x\in A) \\
							g(y+1) &= \mu x(x>g(y)~\&~x\in A).
						\end{align*}
						Weil $A$ rekursiv und nicht endlich ist, ist $g$ rekursiv.
						Offensichtlich ist $g$ streng monoton und $\text{Wb}(g)=A$.
					\item[(iv)$\Rightarrow$(iii)] Falls $A$ nicht endlich ist, genügt die Funktion $g$ aus (iv) unseren Anforderungen.
						Ist $A$ endlich, aber nicht leer, so definieren wir
						\[g(m)=\begin{cases}
							x_m, &\text{falls }m\leq n \\
							x_n, &\text{sonst,}
						\end{cases}\]
						wobei $A=\{x_0,\dots,x_n\}$ mit $x_0<\ldots<x_n$ gelten soll.
						Dann ist $g$ rekursiv, da $g$ über endlich viele rekursive Fallunterscheidungen definiert ist.
					\item[(iii)$\Rightarrow$(ii)] Wegen $A=\text{Wb}(g)$ ist $A$ rekursiv aufzählbar.
						Falls $A$ endlich ist, ist $A$ rekursiv und somit $\overline{A}$ rekursiv aufzählbar. Ansonsten ist $\overline{A}$ wegen
						\[\overline{A}=\{x\in\mathbb{N}\colon\exists y~(g(y)>x~\&~\forall y'<y~(g(y')\neq x)\}\]
						die Projektion einer rekursiven Menge und somit rekursiv aufzählbar.
					\item[(ii)$\Rightarrow$(i)] Siehe Vorlesung.
				\end{enumerate}
			\item \begin{enumerate}[(i)]
					\item Folgt direkt aus (a), da jede streng monotone Funktion injektiv ist.
					\item Da $A$ nicht leer ist, ist $A$ der Wertebereich einer total rekursiven Funktion $f$.
						Dann ist die rekursive Funktion $g$ mit
						\begin{align*}
							g(0) &= f(0) \\
							g(y+1) &= \begin{cases}
								f(y+1), &\text{falls }f(y+1)\geq f(y) \\
								f(y), &\text{sonst}
							\end{cases}
						\end{align*}
						schwach monoton und deshalb ist nach (a) die Menge $B=\text{Wb}(g)$ rekursiv.
						Offensichtlich gilt $\text{Wb}(g)\subseteq \text{Wb}(f)=A$ und weil $A$ unendlich ist,
						gibt es für jedes $y$ ein $y'$ mit $f(y')>g(y)$, also auch $g(y')>g(y)$.
				\end{enumerate}
		\end{enumerate}
	\end{solution}
	
	\begin{question}[subtitle={Klausur 2009}]
		Zeige, dass es zu jeder rekursiv aufzählbaren Menge $A\subseteq\mathbb{N}$ eine 1-stellige partiell rekursive Funktion $\psi$ gibt,
		sodass $A$ sowohl Definitionsbereich als auch Wertebereich von $\psi$ ist: $A=\text{Db}(\psi)=\text{Wb}(\psi)$.
	\end{question}
	\begin{solution}
		Da $A$ rekursiv aufzählbar ist, gibt es eine 1-stellige partiell rekursive Funktion $\theta$,
		deren Definitionsbereich $A$ ist: $A=\text{Db}(\theta)$.
		Definiere die partiell rekursive Funktion $\psi$ durch
		\[\psi(x)=0\cdot\theta(x)+x.\]
		Dann gilt $A=\text{Db}(\theta)=\text{Db}(\psi)=\text{Wb}(\psi)$.
	\end{solution}
	
	\begin{question}[subtitle={Klausur 2009}]
		Es sei $A\subseteq\mathbb{N}$ unendlich und $f:\mathbb{N}\to\mathbb{N}$ total rekursiv.
		Weiter sei $f(A)=\{f(x)\colon x\in A\}$ das Bild und $f^{-1}(A)=\{x\colon f(x)\in A\}$ das Urbild von $A$ bzgl. $f$.
		Welche der folgenden Aussagen gelten stets?
		Begründe deine Antworten! Gebe gegebenenfalls ein Gegenbeispiel an.
		\begin{enumerate}[(i)]
			\item Ist $A$ nicht rekursiv, so ist auch das Bild $f(A)$ von $A$ nicht rekursiv.
			\item Ist $A$ nicht rekursiv und $f$ injektiv, so ist auch $f(A)$ nicht rekursiv.
			\item Ist $A$ rekursiv aufzählbar, so ist auch $f(A)$ rekursiv aufzählbar.
			\item Ist $A$ rekursiv, so ist auch $f(A)$ rekursiv.
			\item Ist $A$ rekursiv, so ist auch $f^{-1}(A)$ rekursiv.
		\end{enumerate}
	\end{question}
	\begin{solution}
		\begin{enumerate}[(i)]
			\item Falsch. Sei $A$ nicht rekursiv. Dann ist für $f(x)=0$ konstant $f(A)=\{0\}$ und damit rekursiv.
			\item Wahr. Da $f$ injektiv ist, gilt $A\leq_mf(A)$ via $f$, denn
				\[x\in A\Rightarrow f(x)\in A\]
				und
				\[x\notin A\Rightarrow f(x')\neq f(x)~\forall x'\Rightarrow f(x)\notin f(A).\]
			\item Wahr. Gilt $A=\text{Wb}(\psi)$ für eine partiell rekursive Funktion $\psi$, so ist $f(A)=\text{Wb}(f\circ\psi)$.
			\item Falsch. Das Halteproblem $K$ ist rekursiv aufzählbar.
				Also gibt es eine total rekursive Funktion $f$ mit $\text{Wb}(f)=K$.
				Dann ist $f(\mathbb{N})=K$ nicht rekursiv.
			\item Wahr. Ist $A=\text{Db}(\psi)$ für eine partiell rekursive Funktion $\psi$, so ist $f^{-1}(A)=\text{Db}(\psi\circ f)$.
		\end{enumerate}
	\end{solution}
	
	\section{Reduktionsmethode}
	\begin{question}[subtitle={Nachklausur 2012}]
		\begin{enumerate}[(a)]
			\item Zeige, dass die Mengen
				\[A=\{\langle x,y\rangle\colon\varphi_x(y)\downarrow\text{ und }\varphi_y(x)\downarrow\}\]
				und
				\[B=\{\langle x,y\rangle\colon\varphi_x(y)\uparrow\text{ oder }\varphi_y(x)\uparrow\}\]
				nicht rekursiv sind.
			\item Sind $A$ und $B$ rekursiv aufzählbar?
		\end{enumerate}
	\end{question}
	\begin{solution}
		\begin{enumerate}[(a)]
			\item Es gilt $K_d\leq_m A$ via $f(x)=\langle x,x\rangle$ da
				\[x\in K_d\Leftrightarrow\varphi_x(x)\downarrow\Leftrightarrow
				\varphi_x(x)\downarrow~\&~\varphi_x(x)\downarrow\Leftrightarrow
				f(x)=\langle x,x\rangle\in A\]
				Da das diagonale Halteproblem nicht rekursiv ist,
				ist also nach dem Reduktionslemma auch $A$ nicht rekursiv.
				Da $B=\overline{A}$ und die rekursiven Mengen gegen Komplement abgeschlossen sind,
				ist somit auch $B$ nicht rekursiv.
			\item $A$ ist als Definitionsbereich der partiell rekursiven Funktion
				\[\psi(\langle x,y\rangle)=\varphi(x,y)+\varphi(y,x)\]
				rekursiv aufzählbar.
				Da eine Menge rekursiv ist, falls die Menge selbst und ihr Komplement rekursiv aufzählbar sind,
				kann also $B$ nicht rekursiv aufzählbar sein.
		\end{enumerate}
	\end{solution}
	
	\begin{question}
		Seien FIN und INF die Indizes der endlichen bzw. unendlichen rekursiv aufzählbaren Mengen:
		\begin{align*}
			\text{FIN} &= \{e\in\mathbb{N}\colon W_e\text{ ist endlich}\} \\
			\text{INF} &= \{e\in\mathbb{N}\colon W_e\text{ ist unendlich}\}
		\end{align*}
		Sei TOT die Menge der Indizes aller rekursiven Funktionen:
			\[\text{TOT}=\{e\in\mathbb{N}\colon W_e=\mathbb{N}\}\]
		\textit{Hinweis}: Wie in der Vorlesung definiert,
		ist $W_e$ der Definitionsbereich der $e$-ten partiell rekursiven Funktion,
		also $W_e=\text{Db}(\varphi_e)=\{x\in\mathbb{N}\colon\varphi_e(x)\downarrow\}$.
		\begin{enumerate}[(a)]
			\item Zeige, dass die Menge
				$\text{FIN}\oplus\text{INF}=\{2e\colon e\in\text{FIN}\}\cup\{2e+1\colon e\in\text{INF}\}$
				nicht rekursiv aufzählbar ist.
			\item Zeige, dass das Komplement von TOT auf FIN many-one-reduzierbar ist,
				also dass $\overline{\text{TOT}}\leq_m\text{FIN}$ gilt.
			\item Zeige, dass FIN nicht rekursiv aufzählbar ist. \\
				\textit{Hinweis}: Entweder zeigst du dies direkt,
				oder du zeigst durch geeignete $m$-Reduzierung,
				dass $\overline{\text{TOT}}$ nicht rekursiv aufzählbar ist.
		\end{enumerate}
	\end{question}
	\begin{solution}
		\begin{enumerate}[(a)]
			\item FIN und INF sind als nichttriviale Indexmengen nach dem Satz von Rice nicht rekusiv.
				Da aber $\text{FIN}=\overline{\text{INF}}$ gilt,
				muss nach dem Komplementlemma zumindest eine der Mengen nicht rekursiv aufzählbar sein.
				Aus $\text{FIN}\leq_m\text{FIN}\oplus\text{INF}$ und $\text{INF}\leq_m\text{FIN}\oplus\text{INF}$
				folgern wir mithilfe der Abgeschlossenheit der rekursiv aufzählbaren Mengen nach unten gegen $m$-Reduzierung,
				dass $\text{FIN}\oplus\text{INF}$ nicht rekusiv aufzählbar ist.
			\item Definiere die zweistellige partiell rekursive Funktion $\psi$ durch
				\[\psi(e,x)=\sum_{n=0}^{x}\varphi_e(n).\]
				Dann gibt es, da $\varphi$ eine Gödelnummerierung ist,
				eine rekursive Übersetzungsfunktion $f$ von $\psi$ nach $\varphi$,
				d.h. $\psi(e,x)=\varphi_{f(e)}(x)$ für alle $x\in\mathbb{N}$.
				Somit gilt \begin{align*}
					e\in\overline{\text{TOT}}
					&\Rightarrow e\not\in\text{TOT} \\
					&\Rightarrow\varphi_e(x_0)\uparrow\text{ für ein }x_0\in\mathbb{N} \\
					&\Rightarrow\psi_e(x)\uparrow\text{ für alle }x\geq x_0 \\
					&\Rightarrow W_{f(e)}=\text{Db}(\varphi_{f(e)})=\text{Db}(\psi_e)\text{ ist endlich} \\
					&\Rightarrow f(e)\in\text{FIN} \\
				\end{align*} und \begin{align*}
					e\not\in\overline{\text{TOT}}
					&\Rightarrow e\in\text{TOT} \\
					&\Rightarrow\varphi_e(x)\downarrow\text{ für alle }x\in\mathbb{N} \\
					&\Rightarrow\psi_e(x)\downarrow\text{ für alle }x\in\mathbb{N} \\
					&\Rightarrow W_{f(e)}=\text{Db}(\varphi_{f(e)})=\text{Db}(\psi_e)=\mathbb{N} \text{ ist unendlich} \\
					&\Rightarrow f(e)\in\text{INF}=\overline{\text{FIN}},
				\end{align*}
				d.h. $e\in\overline{\text{TOT}}\Leftrightarrow f(e)\in\text{FIN}$
				und deshalb $\overline{\text{TOT}}\leq_m\text{FIN}$ via $f$.
			\item Es genügt, zu zeigen, dass $\overline{\text{TOT}}$ nicht rekursiv aufzählbar ist,
				denn dann folgt aus der Abgeschlossenheit der rekursiv aufzählbaren Mengen nach unten gegen $m$-Reduzierung,
				dass FIN nicht rekursiv aufzählbar ist.
				Definiere also die zweistellige partiell rekursive Funktion $\theta$ durch
				\[\theta(e,x)=\varphi_e(e).\]
				Dann gibt es, da $\varphi$ eine Gödelnummerierung ist,
				eine rekursive Übersetzungsfunktion $g$ von $\theta$ nach $\varphi$,
				d.h. $\theta(e,x)=\varphi_{g(e)}(x)$ für alle $e,x\in\mathbb{N}$.
				Somit gilt \[
					e\in K_d
					\Leftrightarrow\varphi_e(e)\downarrow
					\Leftrightarrow\theta_e\text{ total}
					\Leftrightarrow\varphi_{g(e)}\text{ total}
					\Leftrightarrow g(e)\in\text{TOT},
				\]
				d.h. $K_d\leq_m\text{TOT}$ bzw. $\overline{K_d}\leq_m\overline{\text{TOT}}$ via $g$.
				Aufgrund der Transitivität von $\leq_m$ folgt $\overline{K_d}\leq_m\text{FIN}$.
				Also wäre mit FIN auch $\overline{K_d}$ rekursiv aufzählbar,
				da die rekursiv aufzählbaren Mengen nach unten gegen $\leq_m$ abgeschlossen sind.
				Dies ergäbe allerdings einen Widerspruch,
				denn dann wären $K_d$ (laut Vorlesung) und $\overline{K_d}$ (nach Annahme) rekursiv aufzählbar und somit wäre $K_d$ rekursiv.
		\end{enumerate}
	\end{solution}
	
	\begin{question}[subtitle={Klausur 2012}]
		Sei
		\[\text{Id}=\{e\in\mathbb{N}\colon\forall n~(\varphi_e(n)=n)\}.\]
		\begin{enumerate}[(a)]
			\item Zeige, dass das Halteproblem $m$-reduzierbar auf Id ist, das heißt, dass $K\leq_m\text{Id}$ gilt.
			\item Gebe für die folgenden Aussagen jeweils an, ob sie wahr oder falsch sind.
				Begründe deine Antworten!
				\begin{enumerate}[(i)]
					\item Id ist rekursiv.
					\item $\overline{\text{Id}}$ ist rekursiv.
					\item $\overline{\text{Id}}$ ist rekursiv aufzählbar.
				\end{enumerate}
		\end{enumerate}
	\end{question}
	\begin{solution}
		\begin{enumerate}[(a)]
			\item Definiere die zweistellige partiell rekursive Funktion $\psi$ durch
				\[\psi(\langle e,x\rangle,y)=0\cdot\varphi_e(x)+y.\]
				Dann gibt es, da $\varphi$ eine Gödelnummerierung ist,
				eine rekursive Übersetzungsfunktion $f$ von $\psi$ nach $\varphi$,
				d.h. $\psi(\langle e,x\rangle,y)=\varphi_{f(\langle e,x\rangle)}(y)$ für alle $e,x,y\in\mathbb{N}$.
				Somit gilt
				\begin{align*}
					\langle e,x\rangle\in K &\Leftrightarrow \varphi_e(x)\downarrow \\
					&\Leftrightarrow \psi(\langle e,x\rangle,y)=y\text{ für alle }y\in\mathbb{N} \\
					&\Leftrightarrow \varphi_{f(\langle e,x\rangle)}(y)=y\text{ für alle }y\in\mathbb{N} \\
					&\Leftrightarrow f(\langle e,x\rangle)\in\text{Id},
				\end{align*}
				d.h. $K\leq_m\text{Id}$ via $f$.
			\item \begin{enumerate}[(i)]
				\item Falsch. Id ist als nichttriviale Indexmenge nach dem Satz von Rice nicht rekursiv.
					Alternativ kann man argumentieren, dass $K$ nicht rekursiv ist und somit nach dem Reduktionslemma Id auch nicht.
				\item Falsch. Eine Menge ist genau dann rekursiv, wenn ihr Komplement rekursiv ist.
				\item Falsch. Aus $K\leq_m\text{Id}$ via $f$ folgt $\overline{K}\leq_m\overline{\text{Id}}$ via $f$.
					Also wäre mit $\overline{\text{Id}}$ auch $\overline{K}$ rekursiv aufzählbar,
					da die rekursiv aufzählbaren Mengen nach unten gegen $\leq_m$ abgeschlossen sind.
					Dies ergäbe allerdings einen Widerspruch,
					denn dann wären $K$ (laut Votlesung) und $\overline{K}$ (nach Annahme) rekursiv aufzählbar und somit wäre $K$ rekursiv.
			\end{enumerate}
		\end{enumerate}
	\end{solution}
	
	\begin{question}[subtitle={Blatt 13, 2015}]
		\begin{enumerate}[(a)]
			\item Sei $W_e=\text{Db}(\varphi_e)$ die $e$-te rekursiv aufzählbare Menge
				und $W=\{e\in\mathbb{N}\colon0,1\in W_e\}$.
				Zeige $K_d\leq_mW$, indem du eine rekursive Funktion $f:\mathbb{N}\to\mathbb{N}$ angibst mit
				\[\forall e\colon e\in K_d\Leftrightarrow f(e)\in W\]
				Ist $W$ rekursiv aufzählbar?
			\item Folgere aus (a), dass $W$ vollständig für die Klasse der rekursiv aufzählbaren Mengen ist.
			\item Sei nun $I=\{e\in\mathbb{N}\colon W_e\text{ ist nichttriviale Indexmenge}\}$.
				Ist $I$ eine Idexmenge? Ist sie nichttrivial?
			\item Wahr oder falsch?
				Begründe deine Antwort und gebe gegebenenfalls Gegenbeispiele an!
				\begin{enumerate}[(i)]
					\item Ist $B$ rekursiv aufzählbar und $A\leq_m B$, so ist auch $A$ rekursiv aufzählbar.
					\item Man kann den kleinsten Index einer partiell rekursiven Funktion bestimmen.
						Genauer: Die Funktion $min$,
						die jedem Index $e$ die kleinste Zahl $e'$ mit $\varphi_e=\varphi_{e'}$ zuordnet, ist rekursiv
					\item Es gibt eine abzählbare Klasse $C$, die keine vollständige Menge besitzt.
				\end{enumerate}
		\end{enumerate}
	\end{question}
	\begin{solution}
		\begin{enumerate}[(a)]
			\item Wir definieren
				\[\psi(e,x)=\varphi(e,e)\]
				und erhalten eine Übersetzungsfunktion $f$ von $\psi$ nach $\varphi$,
				d.h. $\psi(e,x)=\varphi_{f(e)}(x)$ für alle $x\in\mathbb{N}$.
				Dann gilt
				\[e\in K_d\Rightarrow\psi_e\text{ total}\Rightarrow\varphi_{f(e)}\text{ total}
				\Rightarrow W_{f(e)}=\mathbb{N}\Rightarrow f(e)\in W\]
				und
				\[e\notin K_d\Rightarrow\psi_e(x)\uparrow\forall x\in\mathbb{N}
				\Rightarrow\varphi_{f(e)}(x)\uparrow\forall x\in\mathbb{N}
				\Rightarrow W_{f(e)}=\emptyset\Rightarrow f(e)\notin W\]
				und somit $K_d\leq_mW$ via $f$.
				
				Wegen
				\[W=\{e\in\mathbb{N}\colon\exists s~(\varphi_e(0)\text{ und }\varphi_e(1)\text{ terminieren in $\leq s$ Schritten}\}\]
				ist $W$ die Projektion einer rekursiven Menge und somit rekursiv aufzählbar.
			\item $W$ ist wegen der Härte von $K_d$ und der Transitivität von $\leq_m$ auch hart und somit, da rekursiv aufzählbar, vollständig.
			\item Gilt $\varphi_e=\varphi_{e'}$, so folgt $W_e=W_{e'}$.
				Deshalb ist $W_e$ genau dann nichttrivial, wenn $W_{e'}$ nichttrivial ist, und somit ist $I$ eine Indexmenge.
				$I\neq\mathbb{N}$, denn $e\notin I$ für $W_e=\emptyset$.
				$I\neq\emptyset$, denn $M=\{e\in\mathbb{N}\colon W_e\neq\emptyset\}$ ist eine nichttriviale Indexmenge. Wegen
				\[M=\{e\in\mathbb{N}\colon\exists\langle x,s\rangle~(\varphi_e(x)\text{ terminiert in }\leq s\text{ Schritten})\}\]
				ist $M$ rekursiv aufzählbar.
				Deshalb existiert ein $e\in\mathbb{N}$ mit $W_e=M$. Für dieses $e$ gilt $e\in I$.
				Also ist $I$ nichttrivial.
			\item \begin{enumerate}[(i)]
					\item Wahr. Die rekursiv aufzählbaren Mengen sind nach unten gegen $\leq_m$ abgeschlossen.
					\item Falsch. Ansonsten wären alle Indexmengen rekursiv, denn
						\[e\in\text{Ind}(\varphi_{e_0})\Leftrightarrow min(e)=e_0,\]
						also $\text{Ind}(\varphi_{e_0})\leq_m\{e_0\}$ via $min$,
						wobei $e_0$ der minimale Index von $\varphi_{e_0}$ ist.
					\item Wahr. Siehe Blatt 10, Aufgabe 1.
				\end{enumerate}
				
		\end{enumerate}
	\end{solution}
	
	\begin{question}[subtitle={Klausur 2009}]
		\begin{enumerate}[(a)]
			\item Zeige, dass die Menge
				\[A=\{e\colon W_e\not\subseteq\{0,\dots,e\}\}\]
				$m$-hart für die Klasse der rekursiv aufzählbaren Mengen ist.
				
				\textit{Hinweis}: Du kannst hierbei die Vollständigkeit
				der in der Vorlesung eingeführten Varianten des Halteproblems voraussetzen.
			\item Ist die Menge $A$ sogar $m$-vollständig für die Klasse der r.a. Mengen?
				Begründe deine Antwort!
		\end{enumerate}
	\end{question}
	\begin{solution}
		\begin{enumerate}[(a)]
			\item Aufgrund der Transitivität von $\leq_m$ ist $A$ schon hart, falls $K_d\leq_mA$ gilt.
				Wir definieren deshalb die partiell rekursive Funktion $\psi$ durch
				\[\psi(e,x)=\varphi(e,e).\]
				Dann gibt es eine rekursive Übersetzungsfunktion $f$ mit
				$\psi_e=\varphi_{f(e)}\text{ für alle }e\in\mathbb{N}$.
				Somit gilt
				\[e\in K_d\Rightarrow\psi_e\text{ total}\Rightarrow\varphi_{f(e)}\text{ total}
				\Rightarrow W_{f(e)}=\mathbb{N}\Rightarrow f(e)\in A\]
				und
				\[e\notin K_d\Rightarrow\psi_e(x)\uparrow\forall x\Rightarrow\varphi_{f(e)}(x)\uparrow\forall x
				\Rightarrow W_{f(e)}=\emptyset\Rightarrow f(e)\notin A\]
				und somit $K_d\leq_mA$ via $f$.
			\item Wegen
				\[A=\{e\colon\exists\langle x,s\rangle~(x>e~\&~\varphi_e(x)\text{ terminiert in $\leq s$ Schritten})\}\]
				ist $A$ die Projektion einer rekursiven Menge und damit rekursiv aufzählbar.
		\end{enumerate}
	\end{solution}
	
	\begin{question}[subtitle={Klausur 2009}]
		\begin{enumerate}[(a)]
			\item Zeige, dass die Menge
				\[A=\{e\colon\vert W_e\vert\geq2\}\]
				$m$-hart für die Klasse der rekursiv aufzählbaren Mengen ist.
			
				\textit{Hinweis}: Du kannst hierbei die Vollständigkeit der in der Vorlesung eingeführten Varianten des Halteproblems voraussetzen.
			\item Ist die Menge $A$ sogar $m$-vollständig für die Klasse der r.a. Mengen? Begründe deine Antwort!
		\end{enumerate}
	\end{question}
	\begin{solution}
		\begin{enumerate}[(a)]
			\item Aufgrund der Transitivität von $\leq_m$ ist $A$ schon hart, falls $K_d\leq_mA$ gilt.
				Wir definieren deshalb die partiell rekursive Funktion $\psi$ durch
				\[\psi(e,x)=\varphi(e,e).\]
				Dann gibt es eine rekursive Übersetzungsfunktion $f$ mit
				$\psi_e=\varphi_{f(e)}\text{ für alle }e\in\mathbb{N}$.
				Somit gilt
				\[e\in K_d\Rightarrow\psi_e\text{ total}\Rightarrow\varphi_{f(e)}\text{ total}
				\Rightarrow W_{f(e)}=\mathbb{N}\Rightarrow f(e)\in A\]
				und
				\[e\notin K_d\Rightarrow\psi_e(x)\uparrow\forall x\Rightarrow\varphi_{f(e)}(x)\uparrow\forall x
				\Rightarrow W_{f(e)}=\emptyset\Rightarrow f(e)\notin A\]
				und somit $K_d\leq_mA$ via $f$.
			\item Wegen
				\[A=\{e\colon\exists\langle x_1,x_2,s\rangle~(x_1\neq x_2~\&~\varphi_e(x_1),\varphi_e(x_2)\text{ terminieren in $\leq s$ Schritten})\}\]
				ist $A$ die Projektion einer rekursiven Menge und damit rekursiv aufzählbar.
		\end{enumerate}
	\end{solution}
	
	\section{Formale Sprachen}
	\begin{question}
		Zeige, dass die folgenden Sprachen nicht kontextfrei sind:
		\begin{align*}
			L_1 &= \{w\in\{a,b,c\}^*\colon \#_a(w)=\#_b(w)=\#_c(w)\} & \textit{(Klausur 2014)} \\
			L_2 &= \{a^nb^{2n}c^{3n}\colon n\geq1\} & \textit{(Klausur 2015)} \\
			L_3 &= \{0^{n^2}\colon n\geq0\} & \textit{(Nachklausur 2015)}
		\end{align*}
	\end{question}
	
	\begin{question} % http://algo2.iti.kit.edu/3126.php
		Wahr oder falsch?
		\begin{enumerate}[(a)]
			\item Die Sprache $L_1=\{vw\colon v,w\in\{0,1\}^*,\vert v\vert=\vert w\vert\}$ ist regulär.
			\item Die Sprache $L_2=\{w_0\circ\ldots\circ w_n\colon n\in\mathbb{N}, w_k\in\{0,1\}^k~\forall k\in\{0,\dots,n\}\}$ ist kontextfrei.
			\item Sei $L\subseteq\{0,1\}^*$ eine Sprache mit $\vert L\vert=n$ für ein $n\in\mathbb{N}$. Dann ist $L$ regulär.
		\end{enumerate}
	\end{question}
	\begin{solution}
		\begin{enumerate}[(a)]
			\item Wahr. $L_1$ wird von der Grammatik $G=(\{S,T\},\{0,1\},P,S)$ mit den Regeln
				\[P=\{S\rightarrow0T~|~1T~|~\lambda,~T\rightarrow0S~|~1S\}\]
				erzeugt, denn
				\[L_1=\{w\in\{0,1\}^*\colon\vert w\vert=2n\text{ für ein }n\in\mathbb{N}\}.\]
			\item Falsch. Angenommen, $L_2$ sei kontextfrei.
				Dann gibt es ein $p\in\mathbb{N}$ gemäß dem Pumping-Lemma für kontextfreie Sprachen.
				Sei $z=\lambda\circ0\circ00\circ\ldots\circ0^p\in L_2$.
				Wegen $\vert z\vert\geq p$ gibt es dann eine Zerlegung $z=uvwxy$ mit
				\begin{enumerate}[(i)]
					\item $vx\neq\lambda$
					\item $\vert vwx\vert<p$
					\item $uv^nwx^ny\in L_2~\forall n\in\mathbb{N}$
				\end{enumerate}
				Dann gilt wegen (i) und (ii), dass
				\[\sum_{k=0}^{p-1}k=\vert z\vert-p<\vert uwy\vert<\vert z\vert=\sum_{k=0}^{p}k,\]
				weshalb $uwx$ nicht in $L_2$ liegen kann.
				Dies ist allerdings ein Widerspruch zu Aussage (iii) des Pumpinglemmas.
			\item Wahr. $L$ wird von der rechtslinearen Grammatik $G=(\{S\},\{0,1\},P,S)$ mit den Regeln
				\[P=\{S\to w\colon w\in L\}\]
				erzeugt. Dabei ist $P$ wegen $\vert P\vert=\vert L\vert=n$ endlich.
		\end{enumerate}
	\end{solution}
	
	\begin{question}
		Zeige, dass die Sprache
		\[L=\{w\in\{0,1\}^*\colon w=Bin(n)\text{ für ein }n\in\mathbb{N}\}\]
		der Binärzahlen regulär ist.
	\end{question}
	\begin{solution}
		$L$ wird von der Grammatik $G=(\{S,A\},\{0,1\},P,S)$ mit den Regeln
		\[P=\{S\rightarrow0~|~1A,~A\rightarrow0A~|~1A~|~\lambda\}\]
		ezeugt, denn
		\[L=\{0\}\cup\{1w\colon w\in\{0,1\}^*\}\text{.}\]
	\end{solution}
	
	\begin{question}
		Sei $L$ die Sprache über dem Alphabet $\{a,b,c\}$, die genau die Wörter enthält,
		in denen das Teilwort $abc$ genau einmal vorkommt.
		\begin{enumerate}[(a)]
			\item Gebe eine rechtslineare Grammatik an, die $L$ erzeugt.
			\item Gebe einen deterministischen endlichen Automaten an, der $L$ erkennt.
		\end{enumerate}
	\end{question}
	\begin{solution}
		\begin{enumerate}[(a)]
			\item $L$ wird von der Grammatik $G=(\{S,A,B,C,D,E\},\{a,b,c\},P,S)$ mit den Regeln
				\begin{align*}
					P=\{&S\rightarrow aA~|~bS~|~cS, \\
					&A\rightarrow aA~|~bB~|~cS, \\
					&B\rightarrow aA~|~bS~|~cC, \\
					&C\rightarrow aD~|~bC~|~cC~|~\lambda, \\
					&D\rightarrow aD~|~bE~|~cC~|~\lambda, \\
					&E\rightarrow aD~|~bC~|~\lambda\}
				\end{align*}
				erzeugt
			\item \textbf{TODO}.
		\end{enumerate}
	\end{solution}
	
	\begin{question}
		Sei $L$ die reguläre Sprache
		\[L=\{w\in\{0,1,2\}^*\colon\#_0(w)=0\text{ oder }\#_1(w)=0\text{ oder }\#_2(w)=0\}.\]
		Gebe einen deterministischen endlichen Automaten an, der $L$ erkennt.
	\end{question}
	\begin{solution}
		Zunächst beobachtet man, dass
		\[L=\{0,1\}^*\cup\{0,2\}^*\cup\{1,2\}^*\]
		gilt.
		
		\textbf{TODO}
	\end{solution}
	
	\begin{question}[subtitle={Klausur 2012}]
		Es sei $G=(\{S,X,Y,Z\},\{a,b\},P,S)$ die kontextfreie Grammatik mit der Regelmenge
		\[P=\{S\to XY,~X\to ZXb~|~\lambda,~Y\to bY~|~b,~Z\to a~|~\lambda\}.\]
		\begin{enumerate}[(a)]
			\item Gebe die von $G$ erzeugte Sprache $L(G)$ an.
			\item Gebe ein Wort $w\in L(G)$ minimaler Länge an,
				das mehr als einen Herleitungsbaum besitzt,
				und zeichne zwei verschiedene zugehörigen Herleitungsbäume.
			\item Gebe eine kontextfreie Grammatik in Chomsky-Normalform an, die $L(G)$ erzeugt.
		\end{enumerate}
	\end{question}
\end{document}